    \documentclass[12pt, a4paper, oneside, fontset=windows]{ctexbook}
    \usepackage{amsmath, amsthm, amssymb, bm, graphicx, hyperref, mathrsfs}
    \usepackage{tcolorbox}
    \usepackage{tikz}
    \usetikzlibrary{tikzmark, arrows.meta, shapes}
    \usepackage{geometry} % 设置页面边距
    \geometry{a4paper, margin=1in}
    \setlength{\parindent}{0pt} % 段落不缩进
    \setlength{\parskip}{1em} % 段落间距

    \title{{\Huge{泛函分析}}\\2025秋}
    \author{刘明琦}
    \date{\today}
    \linespread{1.3}
    \newtheorem{theorem}{定理}[section]
    \newtheorem{definition}{定义}[section]
    \newtheorem{lemma}[theorem]{引理}
    \newtheorem{corollary}[theorem]{推论}
    \newtheorem{example}{例}[section]
    \newtheorem{proposition}[theorem]{命题}

    \DeclareMathOperator*{\esssup}{ess\,sup}
    \DeclareMathOperator{\st}{\mathrm{s.t.}}
    \DeclareMathOperator{\diff}{\mathrm{i.f.f.}}
    \DeclareMathOperator{\ie}{\mathrm{i.e.}}

    \begin{document}

    \maketitle

    \pagenumbering{roman}
    \setcounter{page}{1}

    \begin{center}
        \Huge\textbf{前言}
    \end{center}~\

    这是笔记的前言部分. 
    ~\\
    \begin{flushright}
        \begin{tabular}{c}
            刘明琦\\
            \today
        \end{tabular}
    \end{flushright}

    \newpage
    \pagenumbering{Roman}
    \setcounter{page}{1}
    \tableofcontents
    \newpage
    \setcounter{page}{1}
    \pagenumbering{arabic}

    \chapter{距离线性空间}

    小结. 

    \section{选择公理、良序定理、Zorn引理}

    {\bf 良序定理}
    \begin{theorem}[超限归纳法]
        对良序集$\mathcal{A}$,如果:
        \begin{enumerate}
            \item $P(\alpha_0)$为真,$\alpha_0$是最小元
            \item 若$P(\alpha)$对一切$\alpha,\alpha_0 \prec \alpha \prec \beta$为真,则$P(\beta)$真
        \end{enumerate}
        则$P(\alpha)$对一切$\alpha \in \mathcal{A}$为真
    \end{theorem}

    \begin{tcolorbox}
        {\bf 选择公理} 对于任何一列集合列$\mathcal{N}=\{N\}$,可以是有限、可数、不可数,都自然存在一个选择函数$f \text{ 定义在 }\mathcal{N},f(N)=n \in N$
    \end{tcolorbox}

    \begin{tcolorbox}
        {\bf 良序定理} 任何一个集合都能赋予先后次序使之成为良序集
    \end{tcolorbox}

    \begin{tcolorbox}
        {\bf Zorn引理}  偏序集 P 中的每一个链 (Chain) 都在 P 中有一个上界。则P 必有极大元。
    \end{tcolorbox}

    \section{线性空间、Hamel基}
    \begin{definition}[线性空间]
        实线性空间、复线性空间
    \end{definition}
    \begin{definition}[线性流形]
        线性空间X的非空子集M:\[\forall x,y \in M , x + y,ax \in M\]
    \end{definition}
    \begin{definition}
        线性相关、维数$\dim $、基、直和
    \end{definition}

    \begin{theorem}
        M,N是线性空间X的线性流形,则:\[X=M\oplus N \Leftrightarrow \forall x \in X \text{ 唯一表为 } x=m+n ,m\in M,n \in N\]
        称M,N为\textbf{代数互补}
    \end{theorem}

    \begin{definition}[Hamel基]
        X是有非零元的线性空间,$H \subset X$是Hamel基:
        \begin{enumerate}
            \item H 线性无关
            \item H张成的线性流形是X
        \end{enumerate}
    \end{definition}
    \begin{theorem}
        X有覆盖任意线性无关子集的Hamel基: $\exists H,s.t. S \subset H$
    \end{theorem}

    \begin{theorem}[代数补的存在性]
        线性空间X的线性流形M必存在代数补N,即$X = M \oplus N$
    \end{theorem}

    \section{距离空间、距离线性空间}
    \begin{definition}[距离空间]
        定义距离d(x,y):
        \begin{enumerate}
            \item d(x,y) $\ge$ 0
            \item d(x,y)=d(y,x)
            \item d(x,z) $\le$ d(x,y)+d(y,z)
        \end{enumerate}
        则称$\langle X,d\rangle$ 为距离空间 
    \end{definition}

    \begin{definition}[依距离收敛]
        $\{x_n\}^{\infty}_{n=1} \subset \langle X,d\rangle,s.t.:$
        \[ \lim_{n \to \infty} d(x_n,x)=0 \text{或者} d(x,x_n) \to 0 \]  
    \end{definition}

    \begin{definition}[距离线性空间]
        线性空间X定义$d(\cdot ,\cdot )$且d的极限对加法和数乘连续       
    \end{definition}

    \begin{definition}[本质上确界]
        $f(t)\in L([a,b])$ 即区间上的勒贝格可测函数,如果存在零测$E \subset [a,b]$,s.t.f(t)在$[a,b]\backslash E$上有界,则称f是$[a,b]$上本质有界函数,定义\textbf{本质上确界}为:
        \[ \esssup_{t\in [a,b]}|f(t)|=\inf_{m(E)=0}\{\sup_{t\in [a,b]\backslash E}|f(t)|\}\]
        
    \end{definition}
    下面是几个常见空间
    \begin{enumerate}
        \item {\bf 本质有界可测函数空间 $L^{\infty}[a,b]$}:
        \[
        \begin{cases}
            \displaystyle X=\{f(t),t\in [a,b]:\ f\ \text{本质有界}\}\\
            \displaystyle d(x,y)=\esssup_{t\in [a,b]}:|x(t)-y(t)|
        \end{cases}
        \]
        \item {\bf 有界序列空间(m)或$l^{\infty}$}:
        \[ \begin{cases}
            \displaystyle X=\{x|x=\{\xi_1,\xi_2,...\xi_j...\}\}\\
            \displaystyle d(x,y)=\sup_{j\ge 1} |\xi _j - \eta _j |
            \end{cases}\]

        \item {\bf 收敛序列空间(c)}: 
        \[
            \begin{cases}
                \displaystyle X=\{x:\ x=\{\xi_j\},\ \lim_{j\to \infty}\xi_j=\xi  < \infty\}\\
                \displaystyle d(x,y)=\sup_{j\ge 1} |\xi _j - \eta _j |
            \end{cases}
        \]
        \item {\bf 所有序列空间(s)}: 
        \[
            \begin{cases}
                \displaystyle X=\{x:\ x=\{\xi_j\}\}\\
                \displaystyle d(x,y) = \displaystyle \sum_{j=1}^{\infty } \dfrac{1}{2^j} \dfrac{|\xi_j - \eta_j|}{1+|\xi_j -\eta_j|}
            \end{cases}
        \]
        \item {\bf p-可和数列空间/p-可积函数空间}
    \end{enumerate}
        
    \begin{table}[h]
        \centering
        \begin{tabular}{|c|c|c|}
        \hline
        & $l^p$ 空间 & $L^p[a,b]$ 空间 \\
        \hline
        元素 & 数列 $x=(\xi_1,\xi_2,\dots)$ & 可测函数 $x(t)$,定义在区间 $[a,b]$ 上 \\
        \hline
        条件 & $\displaystyle \sum_{i=1}^\infty |\xi_i|^p < \infty$ & $\displaystyle \int_a^b |x(t)|^p \, dt < \infty$ \\
        \hline
        范数 & $\displaystyle \|x\|_p = \Big(\sum_{i=1}^\infty |\xi_i|^p\Big)^{1/p}$ & $\displaystyle \|x\|_p = \Big(\int_a^b |x(t)|^p \, dt\Big)^{1/p}$ \\
        \hline
        距离 & d(x,y)=$\displaystyle (\sum_{j=1}^{\infty}|\xi_j - \eta_j|^p)^{1/ p}$ & d(x,y)=$\displaystyle (\int_{a}^{b}|x(t)-y(t)|^p \mathrm{d}t )^{1/p}$\\
        \hline
        本质 & 离散型的 $p$-可和数列空间 & 连续型的 $p$-可积函数空间 \\
        \hline
        直观类比 & “在整数点上的取值” & “在连续区间上的取值” \\
        \hline
        空间性质 & Banach 空间;$p=2$ 时是 Hilbert 空间 & Banach 空间;$p=2$ 时是 Hilbert 空间 \\
        \hline
        \end{tabular}
    \end{table}


    \begin{tcolorbox}
        \textbf{Minkowski不等式}\\
        离散形式(n有限或=$\infty$):
        \[\left( \sum_{i=1}^{n} |x_i + y_i|^p \right)^{1/p} \le \left( \sum_{i=1}^{n} |x_i|^p \right)^{1/p} + \left( \sum_{i=1}^{n} |y_i|^p \right)^{1/p}\] 
            用范数表示:
            \[ \|x+y\|_p \le \|x\|_p + \|y\|_p\]
        积分形式:
        \[\left( \int_S |f(x) + g(x)|^p \,d\mu \right)^{1/p} \le \left( \int_S |f(x)|^p \,d\mu \right)^{1/p} + \left( \int_S |g(x)|^p \,d\mu \right)^{1/p}\]
            范数表示:
            \[\|f+g\|_{L^p(S)} \le \|f\|_{L^p(S)} + \|g\|_{L^p(S)}\]
    \end{tcolorbox}

    \section{距离空间中的拓扑、可分空间}
    \begin{definition}
        球$B(x_0,r)$、开集、闭集、极限点、内点、内部、连续函数
    \end{definition}

    \begin{definition}
        $y=f(x):\langle X,d\rangle \rightarrow \langle Y,\rho \rangle $\\
        f将邻域映为邻域则称之为连续
    \end{definition}

    \begin{definition}[稠密集]
        $S\subset X,\ \forall \varepsilon:\ \forall x \in X,\exists x_0 \in S,\st \ d(x,x_0)<\varepsilon$\\
        $\mathrm{eg:}\mathbb{Q}\subset \mathbb{R}$
    \end{definition}

    \begin{definition}[可分空间]
    一个拓扑空间 $X$ 被称为是\textbf{可分的 (separable)},如果它包含一个\textbf{可数}且\textbf{稠密}的子集。也就是说,存在一个子集 $A \subseteq X$,满足:
    \begin{enumerate}
        \item $A$ 是可数的 (countable),即 $A$ 中的元素可以与自然数集 $\mathbb{N}$ 建立一一对应关系。
        \item $A$ 在 $X$ 中是稠密的 (dense),即 $A$ 的闭包 $\overline{A}$ 等于全空间 $X$。换句话说,对于 $X$ 中任何一点 $x$ 以及 $x$ 的任何一个邻域 $U$,都有 $U \cap A \neq \emptyset$。
    \end{enumerate}
    直观上讲,一个空间是可分的,意味着我们可以用一个可数无限的点集来“近似”整个空间中的任意一点。
\end{definition}

下面是一些常见可分与不可分空间的例子:
\begin{example}
    \begin{itemize}
        \item \textbf{实数空间 $\mathbb{R}$} 是可分的。有理数集 $\mathbb{Q}$ 是 $\mathbb{R}$ 的一个可数稠密子集。
        \item \textbf{欧几里得空间 $\mathbb{R}^n$} 是可分的。所有分量都是有理数的点的集合 $\mathbb{Q}^n$ 是一个可数稠密子集。
        \item \textbf{收敛序列空间 (c)}、\textbf{所有序列空间 (s)} 和 \textbf{$l^p$ 空间 ($1 \le p < \infty$)} 都是可分的。
        \item \textbf{连续函数空间 $C[a, b]$} (在 $[a, b]$ 上所有连续函数构成的空间,赋予上确界范数) 是可分的。根据Weierstrass逼近定理,所有系数为有理数的多项式函数构成的集合是一个可数稠密集。
        \item \textbf{有界序列空间 $l^{\infty}$ (或(m))} 是\textbf{不可分}的。这个空间太“大”了,无法用一个可数子集来逼近所有点。
    \end{itemize}
\end{example}

可分性在泛函分析中有很多重要的推论,例如在可分赋范线性空间中,单位球上的弱*拓扑是可度量化的。

    \section{完备的距离空间}

    \begin{definition}
        Cauthy序列
    \end{definition}

    $\cdot$ 收敛一定是柯西列,柯西列不一定收敛\\

    eg:在有理数空间 $\langle \mathbb{Q}, d(x, y) = |x - y| \rangle$ 中逼近 $\sqrt{2}$

    \textbf{构造柯西列 (Constructing the Cauchy Sequence):} \\
    我们构造一个在实数意义下收敛到 $\sqrt{2}$ 的有理数序列 $\{x_n\}$。一个简单的方法是取 $\sqrt{2}$ 的十进制小数展开:
    $$ \sqrt{2} = 1.41421356\dots $$
    我们定义序列如下:
    $$ x_1 = 1, \quad x_2 = 1.4, \quad x_3 = 1.41, \quad x_4 = 1.414, \quad \dots $$


    \begin{definition}[\bf 完备空间]
        \ \\距离空间$\langle X,d \rangle$中所有Cauchy序列收敛则称他是\textbf{完备的}
    \end{definition}

    \begin{definition}[距离空间的完备化]
        对距离空间$\langle X,d \rangle$,如果有完备的距离空间$\langle \widetilde{X}, \rho \rangle\ $使X等距于$\widetilde{X} $的稠密子集,即存在映射$T:X\to \widetilde{X},\ \st $
        \[
            d(x,y)=\rho (T(x),T(y))
        \]
        且T(X)是$\widetilde{X} $的稠密子集,则称$\widetilde{X} $是X的\textbf{完备化}
    \end{definition}

    \begin{theorem}
        任何距离空间都存在完备化
    \end{theorem}


    \section{列紧性}
    \begin{theorem}
        直线上每个有界的无穷点集至少有一个聚点
    \end{theorem}

    \begin{definition}[列紧性]
        距离空间X中的集合M称为\textbf{列紧的},如果M中的任何序列都含有一个收敛的子序列(其极限未必还在M中).闭的列紧集成为\textbf{自列紧集}
    \end{definition}

    \begin{definition}
        距离空间X中的集合M称为\textbf{完全有界}的,如果任意给定的$\varepsilon > 0$,总存在由有限个元组成的M的$\varepsilon$-网\\
        $\varepsilon$-网:$\forall \varepsilon > 0:\ \forall x \in M,\ \exists x' \in N,\st d(x,x')<\varepsilon$,称N是M的$\varepsilon$-网
    \end{definition}

    \begin{theorem}
        列紧性蕴含完全有界性,对于完备的空间X而言,列紧性等价于完全有界性。
    \end{theorem}

    \begin{theorem}
        距离空间中的任何完全有界集是可分的
    \end{theorem}

    \begin{definition}
        紧集:任何开覆盖存在有限子覆盖
    \end{definition}

    \begin{theorem}
        距离空间中紧性和自列紧性等价
    \end{theorem}

    \begin{definition}[同等连续]
        $\mathcal{F}$是一族距离空间$\langle X,d \rangle$到$\langle Y,\rho \rangle$的函数,如果任意给定$\varepsilon > 0$,存在$\delta > 0$,使得对任意$f\in \mathcal{F}$都有
        \[
            \rho(f(x),f(x'))<\varepsilon,\text{当}d(x,x')<\delta
        \]
        则称$\mathcal{F}$是同等连续的
    \end{definition}

    \begin{tcolorbox}
        \textbf{对角线方法}(分析学常用证明紧性)\\
        设有多列有界数列$\{\alpha_{kn}\}_{n=1}^{\infty}$那么每列都有自己的收敛子列$\{\alpha_{km}\}$下标列$\{km\}$是不同的指标序列,那么我们的目标就是找一列共同的\textbf{下标列},对每一列数列都适用
    \end{tcolorbox}

    \section{赋范线性空间}
    \begin{definition}[赋范线性空间]
        对复的或实的线性空间X,若有X到$\mathcal{R}$的函数$\| x \Vert , \st$
        \begin{enumerate}
            \item 非负定: $||x||=0 \iff x=0$
            \item 线性性: $||ax||=|a|\cdot ||x||$
            \item 三角不等式
        \end{enumerate}
        即:线性空间X赋予范数$||\cdot ||$满足三条件,记为$\langle X,||\cdot || \rangle$
    \end{definition}

    赋范线性空间X中,$||x||$是$x\in X$的连续函数(依赖于收敛的定义)\\
    \begin{definition}[线性算子]
        \[T:\langle X,||\cdot ||_1\rangle \to \langle Y,||\cdot||_2 \rangle,\ \forall x,y\in X,\alpha,\beta \text{是数,都有:}\]
        \[
        T(\alpha x+\beta y)=\alpha Tx+\beta Ty\]
        如果Tx有界,即$||Tx||_2 \le C||x||_1$,定义\textbf{线性算子T的范数${\bf ||T||}$}:
        \[
            ||T||=\sup_{||x||_1=1}||Tx||_2=\sup \{\dfrac{||Tx||_2}{||x||_1}\}
        \]
    \end{definition}
    \begin{tcolorbox}
        线性算子的范数实际就是他把单位元线性映射后的最小像大小
    \end{tcolorbox}
    \begin{theorem}[性质]
        下列几条结论相互等价
        \begin{enumerate}
            \item T在X中某点连续
            \item T在X中所有点连续
            \item T有界
        \end{enumerate}
    \end{theorem}
    \begin{definition}[线性泛函]
        线性空间X上的复值函数$f:X\to \mathbb{C}$称为线性泛函,如果$\forall x,y\in X,\text{数}\alpha ,\beta $有\[
        f(\alpha x+\beta y)=\alpha f(x)+\beta f(y)
        \]
    \end{definition}

    \begin{tcolorbox}[title code={123}]
        $x_1,\dots,x_n$是赋范线性空间X中线性无关的元素,则有$\mu >0,\st$
        \[|\alpha_1|+\dots +|\alpha_n| \le \ \mu \  ||\alpha_1 x1+\dots +\alpha_n x_n||\]
    \end{tcolorbox}

    \begin{definition}
        \ \\
        \textbf{子空间:}距离线性空间中,闭的线性流形称为\textbf{子空间}\\
        \textbf{线性流形:}线性空间X的非空子集M:$\forall x,y \in M , x + y,ax \in M$
    \end{definition}

    \begin{theorem}[\textbf{Reisz}]
        赋范线性空间X的真子空间M,则对任意给定$\varepsilon >0$,存在$x_\varepsilon >0,\st ||x_{\varepsilon}||=1$,且    
        \[
        \rho (x_{\varepsilon},M)=\inf_{m\in M}||x_{\varepsilon}-m|| \ge \ 1-\varepsilon
        \]
    \end{theorem}

    完备的赋范线性空间称为\textbf{巴拿赫空间(Banach Space)}\\

    \section{压缩映像原理}
    \begin{definition}[Lipschitz条件]
        设$\langle X,d \rangle$和$\langle Y,\rho \rangle$是两个度量空间,映射$T:X\to Y$满足
        \[
            \rho (T(x),T(y))\le K d(x,y),\ \forall x,y\in X
        \]
        则称$T$是\textbf{Lipschitz连续}的,$K$为\textbf{Lipschitz常数}
    \end{definition}
    特别的,如果$K<1$,则称$T$为\textbf{压缩映像},$K$为\textbf{压缩常数}\\
    如果$Tx=x$,则称$x$为$T$的\textbf{不动点}\\
    \begin{theorem}
        距离空间X中符合Lipschitz条件的映射T是连续的
    \end{theorem}
    下面正式给出{\bf 压缩映像原理}
    \begin{theorem}[压缩映像原理]
        设$\langle X,d \rangle$是完备的距离空间,$T:X\to X$是X到自身的压缩映像,则T有唯一不动点$\bar{x}\in X$,即$T\bar{x}=\bar{x},\bar{x}$有如下性质:\\
        (1) 对任意$x_0\in X$,由$x_{n+1}=Tx_n$所生成的序列$\{x_n\}$收敛于$\bar{x}$\\
        (2) $d(x_n,\bar{x})\le \dfrac{K^n}{1-K}d(x_0,Tx_0)$:对收敛速度的估计
    \end{theorem}
    性质2也就是在说这种压缩不动点是一个很强的收敛点,收敛速度几乎强于几何级数\\
    可以利用压缩映像原理证明{\bf Picard定理},即常微分方程初值问题的解的存在唯一性:\\
    \begin{example}[Picard定理]
        f(t,x)在区域$D=\{(t,x):|t-t_0|\le a,|x-x_0|\le b\}$上连续且对x满足Lipschitz条件,则初值问题
        \[\dfrac{dx}{dt}=f(t,x),\ x(t_0)=x_0\]
        在区间$[t_0-h,t_0+h],\ h=\min\{a,\dfrac{b}{M}\},\ M=\max_{D}|f(t,x)|$上有唯一解
    \end{example}
    \begin{example}[隐函数存在定理]
        设 $x^0 = (x_1^0, \dots, x_n^0) \in \mathbb{R}^n$, $y^0 = (y_1^0, \dots, y_m^0) \in \mathbb{R}^m$。
        设 $U \times V \subset \mathbb{R}^n \times \mathbb{R}^m$ 是 $(x^0, y^0)$ 的一个邻域,$f: U \times V \to \mathbb{R}^m$ 是一个连续函数,并且 $f$ 关于 $y = (y_1, \dots, y_m)$ 的所有偏导数在 $U \times V$ 上都连续。
        如果满足以下条件:
        \begin{enumerate}
            \item $f(x^0, y^0) = 0$
            \item $\det\left[\dfrac{\partial f}{\partial y}(x^0, y^0)\right] \neq 0$
        \end{enumerate}
        则存在 $x^0$ 的一个邻域 $U_0 \subset U$ 以及唯一的连续函数 $\varphi: U_0 \to \mathbb{R}^m$,使得:
        \[
            \begin{cases}
                f(x, \varphi(x)) = 0, & \forall x \in U_0 \\
                \varphi(x^0) = y^0
            \end{cases}
        \]
        其中,$\dfrac{\partial f}{\partial y}(x^0, y^0)$ 是 $f(x,y)$ 作为 $y$ 的函数在 $y^0$ 处的雅可比矩阵:
        \[
            \dfrac{\partial f}{\partial y}(x^0, y^0) = 
            \begin{pmatrix}
                \dfrac{\partial f_1}{\partial y_1}(x^0, y^0) & \cdots & \dfrac{\partial f_1}{\partial y_m}(x^0, y^0) \\
                \vdots & \ddots & \vdots \\
                \dfrac{\partial f_m}{\partial y_1}(x^0, y^0) & \cdots & \dfrac{\partial f_m}{\partial y_m}(x^0, y^0)
            \end{pmatrix}
        \]
        而 $\det[\cdot]$ 表示其行列式。
    \end{example}

    压缩映像原理可以用来证明许多分析学中的重要定理.\\

    下面介绍{\bf Fréchet导数}
    \begin{definition}[Fréchet导数]
        设X,Y是Banach空间,$F:X\to Y$,如果存在从X到Y的有界线性算子$A$,使得
        \[
            \lim_{||h||\to 0}\dfrac{||F(x+h)-F(x)-Ah||_Y}{||h||_X}=0
        \]
        则称F在点x处\textbf{Fréchet可微},A为F在x处的\textbf{Fréchet导数},记为$F'(x)$
    \end{definition}

    \begin{tcolorbox}
        “由于F'(x)定义为映像的线性主部,所以正好反映了将非线性问题线性化,他是应用的最多的一种微分概念”
    \end{tcolorbox}

    \chapter{Hilbert空间}
    \section{内积空间}
    \begin{definition}[内积空间]
        复线性空间X,如果X上定义了内积$(\cdot ,\cdot ):\forall x,y\in X,\exists !(x,y) \in \mathbb{C}$,满足:
        \begin{enumerate}
            \item $(x,x)\ge 0,(x,x)=0 \  \iff \  x=0$
            \item $(\alpha x+y,z)=\alpha (x,z)+(y,z),\ \forall x,y,z\in X,\ \forall \alpha \in \mathbb{C}$\quad \underline{注意:$(x,ay)=\bar{a}(x,y)$}
            \item $(x,y)=\overline{(y,x)},\ \forall x,y\in X$
        \end{enumerate}
        则称X为\textbf{内积空间}。
    \end{definition}

    X中的一族元素$\{x_j\}$称为\textbf{正规正交集},如果对任意的$i\neq j$,都有$(x_i,x_j)=0$。书上用了一个简单的记号:Kronecker-$\delta$:
    \[
        \delta_{ij}=
        \begin{cases}
            1, & i=j \\
            0, & i\neq j
        \end{cases}
    \]
    自然的,可以定义这个内积空间中的范数为:
    \[||x||={(x,x)}^{1/2}\]
    不难验证他满足我们的范数公设,所以这样定义出来的内积空间X也是一个赋范线性空间\\
    这样,我们可以顺带得到一个有趣的结论:$x,y$正交,则$||x+y||^2=||x||^2+||y||^2$,这就是\textbf{勾股定理}的推广。由此,我们可以利用正规正交集把任意向量分解:\\

    \begin{theorem}[内积空间中的勾股定理]
        设 $\{x_n\}_{n=1}^N$ 是内积空间 $X$ 中的正规正交集,则对任何 $x \in X$ 都有
        \[
            \|x\|^2 = \sum_{n=1}^N |( x, x_n )|^2 + \left\|x - \sum_{n=1}^N ( x, x_n ) x_n\right\|^2.
        \]
    \end{theorem}
    \begin{tcolorbox}
        在这里其实说的就是x的正交分解:
        \[
            x = \tikzmarknode{proj}{\sum_{n=1}^N ( x, x_n ) x_n} + \tikzmarknode{ortho}{\left(x - \sum_{n=1}^N ( x, x_n ) x_n\right)}
        \]
        \begin{tikzpicture}[overlay, remember picture, >=Stealth]
            \draw[->, black, thick] (proj.south) -- ++(0,-0.5) node[below, align=center, text width=3cm, yshift=-2mm] {$x$ 在 $\{x_n\}$ \\所张成子空间上的投影 \\ $x_{\text{proj}}$};
            \draw[->, black, thick] (ortho.south) -- ++(0,-0.5) node[below, align=center, text width=3cm, yshift=-2mm] {与该子空间\\正交的分量 \\ $x_{\text{ortho}}$};
        \end{tikzpicture}
        \vspace{3cm} % 留出空间给箭头
        
    \end{tcolorbox}

    \begin{corollary}[Bessel 不等式]
        设 $\{x_n\}_{n=1}^N$ 是内积空间 $X$ 中的正规正交集,则对任何 $x \in X$ 都有
        \[
            \sum_{n=1}^N |( x, x_n )|^2 \le \|x\|^2.
        \]
        \begin{center}
            \text{直角三角形斜边大于直角边}
        \end{center}
    \end{corollary}

    \begin{corollary}[Schwarz 不等式]
        对内积空间 $X$ 中任意两个向量 $x, y$ 都有
        \[
            |( x, y )| \le \|x\| \|y\|.
        \]
    \end{corollary}
    现在我们回到对空间的讨论上来,我们先尝试证明按照内积定义的范数满足范数公设,从而说明{\bf 赋范线性空间}:
    \begin{tcolorbox}
        范数公设:
        \begin{enumerate}
            \item 非负定: $||x||=0 \iff x=0$
            \item 线性性: $||ax||=|a|\cdot ||x||$
            \item 三角不等式
        \end{enumerate}
    \end{tcolorbox}
    \begin{proof}
        公设1、2根据内积的定义不难验证,下面证明公设3.$\forall x,y \in X$\\
        \begin{align*}
            ||x+y||^2 &= (x+y, x+y) \\
                        &= (x,x) + (x,y) + (y,x) + (y,y) \\
                        &= \|x\|^2 + 2\mathrm{Re}(x,y) + \|y\|^2 \\
                        &\le \|x\|^2 + 2|(x,y)| + \|y\|^2 \\
                        &\le \|x\|^2 + 2\|x\|\|y\| + \|y\|^2 \quad (\text{Schwarz不等式})\\
                        &= (\|x\|+\|y\|)^2.
        \end{align*}
        两边开根号即可得证。
    \end{proof}

    \begin{proposition}
        内积$(x,y)$在$X$中对$x,y$连续, \ $\ie n\to \infty:$
        \[
            x_n \to x,\ y_n \to y \Rightarrow (x_n,y_n) \to (x,y)
        \]
    \end{proposition}
    \begin{proposition}
        内积空间X的稠密子集M,若有$x_0\in X,\st$
        \[ (x_0,x)=0,\ \forall\  x\in M\]
        则$x_0=0$
    \end{proposition}
    \begin{theorem}[极化恒等式]
        设X是内积空间,则$\forall x,y\in X$,有
        \begin{align*}
            (x,y)&=\dfrac{1}{4}\sum_{k=0}^3 i^k ||x+i^k y||^2\\
            & = \dfrac{1}{4}(||x+y||^2 - ||x-y||^2 + i||x+iy||^2 - i||x-iy||^2)
        \end{align*}
    \end{theorem}
    \begin{tcolorbox}
        在实线性空间中就是:
        \begin{align*}
                \alpha \cdot \beta &=\dfrac{1}{4}\left[ ( \alpha + \beta ) ^2 - ( \alpha - \beta ) ^2\right]\\
                & = \left(\dfrac{\alpha + \beta}{2}\right)^2 - \left(\dfrac{\alpha - \beta}{2}\right)^2
        \end{align*}
        
    \end{tcolorbox}
    可以顺便得到一个有趣的结论:\\
    \begin{corollary}[平行四边形法则]
        $\forall x,y$是X中的向量
        \[
            ||x+y||^2 + ||x-y||^2 = 2(||x||^2 + ||y||^2)
        \]
        \begin{center}
            \text{平行四边形对角线平方和等于两倍邻边平方和。}
        \end{center}
    \end{corollary}

    内积空间变成赋范线性空间是一个很平凡的过程,因为上述的所有讨论本质还是在初等Euchlid空间中的扩展,很多性质和概念可以直接类比过去.\\
    问题在于:是否对于所有的赋范线性空间都能按照这种构造将距离范数$||x||$表为$[(x,x)]^{1/2}?$\\
    答案是否定的。但是可以有一个较弱的结论:
    {\bf 
    \begin{center}
        X能赋以内积的充要条件是X中的范数满足平行四边形法则
    \end{center}
    }
    \begin{example}
        在空间$C[0,1]$中,取$x(t)=1,y(t)=t:$
        \[
            ||x+y||=\max_{0 \le t \le 1}|1+t|=2\\
            ||x-y||=\max_{0 \le t \le 1}|1-t|=1\\
            ||x||=||y||=1
        \]
        不满足平行四边形法则,所以$C[0,1]$不是内积空间
    \end{example}

    现在,我们可以正式给出Hilbert空间的定义:\\
    \begin{definition}[Hilbert空间]
        \underline{完备的内积空间}H叫做{\bf Hilbert空间}
    \end{definition}
    \begin{tcolorbox}
        距离空间$\langle X,d \rangle$中所有Cauchy序列收敛则称他是\textbf{完备的}\\
        
        \begin{center}
    \begin{tikzpicture}[
        scale=1.2,
        every node/.style={text centered, text width=3cm},
        level/.style={draw, ellipse, minimum height=1.5cm, minimum width=2.5cm, thick},
        arrow/.style={->, >=Stealth, thick}
    ]
        % Define nodes for spaces
        \node[level, minimum width=10cm, minimum height=6.5cm, label={[xshift=4.5cm, yshift=2.8cm]距离空间}] (metric) {};
        \node[level, minimum width=8cm, minimum height=5cm, label={[xshift=3.5cm, yshift=2.1cm]赋范线性空间}] (normed) at (0, -0.5) {};
        \node[level, minimum width=6cm, minimum height=3.5cm, label={[xshift=2.5cm, yshift=1.4cm]Banach空间}] (banach) at (0, -1) {};
        \node[level, minimum width=4cm, minimum height=2.5cm, label={[xshift=1.5cm, yshift=0.9cm]Hilbert空间}] (hilbert) at (0, -1.5) {};
        \node[level, minimum width=6cm, minimum height=2.5cm, label={[xshift=-2.5cm, yshift=0.9cm]内积空间}] (inner) at (0, -1.5) {};

        % Draw relationships
        \node at (0, 2.5) {\textbf{空间关系图}};
        
        \node[text width=2cm] at (2.5, -0.2) {完备化};
        \draw[arrow] (normed.north) to[bend left=20] node[midway, right, xshift=2mm] {} (banach.north);
        
        \node[text width=2cm] at (2.5, -1.8) {完备化};
        \draw[arrow] (inner.north) to[bend left=20] node[midway, right, xshift=2mm] {} (hilbert.north);
        
        \node[text width=3.5cm, align=center] at (-3, -0.5) {范数满足\\平行四边形法则};
        \draw[arrow] (banach.west) to[bend left=20] node[midway, above] {} (hilbert.west);

    \end{tikzpicture}
\end{center}
    \end{tcolorbox}

    \begin{example}
        考察$l^2$空间:$\displaystyle \sum_{n=1}^{\infty}|\xi _n|^2<\infty$的复序列$\{\xi_n\}_{n=1}^{\infty}$,定义内积为
        \[
            (\xi,\eta)=\sum_{n=1}^{\infty}\xi_n\overline{\eta_n}.
        \]
        可以验证$l^2$空间满足内积公设,则$\underline{l^2 \text{按照}{(\cdot ,\cdot)}\text{是一个内积空间}}$,下证完备\\
        设$\{\xi^{(k)}\}$是$l^2$中的Cauchy列,则对任意$\varepsilon >0$,存在$N>0$,当$m,n>N$时,有
        \[||\xi^{(n)}-\xi^{(m)}||^2=\sum_{j=1}^{\infty}|\xi_j^{(n)}-\xi_j^{(m)}|^2<\varepsilon ^2\]
        则对任意的$j$,有$|\xi_j^{(n)}-\xi_j^{(m)}|<\varepsilon$,所以$\{\xi_j^{(n)}\}$是复数域上的Cauchy列,故存在极限$\xi_j=\lim_{n\to \infty}\xi_j^{(n)}$,从而定义$\xi=\{\xi_j\}$\\
        下面证明$\xi \in l^2$,以及$\xi^{(n)}\to \xi$\\
        由Cauchy列的定义,对任意$\varepsilon >0$,存在$N>0$,当$m,n>N$时,有
        \[\sum_{j=1}^{\infty}|\xi_j^{(n)}-\xi_j^{(m)}|^2<\varepsilon ^2\]   则对任意的$j$,有$|\xi_j^{(n)}-\xi_j^{(m)}|<\varepsilon$,所以$\{\xi_j^{(n)}\}$是复数域上的Cauchy列,故存在极限$\xi_j=\lim_{n\to \infty}\xi_j^{(n)}$,从而定义$\xi=\{\xi_j\}$\\
    \end{example}

    常用的Hilbert空间是函数空间,其中最简单的是$L^2[a,b]$空间:
    \begin{example}
        空间$L^2[a,b]$\\
        有限区间上的复平方可积函数空间$L^2[a,b],f,g\in L^2$,定义内积为
        \[
            (f,g)=\int_a^b f(t)\overline{g(t)}dt.
        \]
        容易验证他是一个内积空间,而完备性就是我们在$\S \  1$中证明过的内容。
    \end{example}

    \begin{proposition}
        内积空间X的完备化$\widetilde{X}$是一个Hilbert空间。
    \end{proposition}

    \section{正规正交基}
    此后我们习惯用H表示非零的Hilbert空间\\
    \begin{tcolorbox}
        对于H中任何一列线性无关的$\{u_n\}$,都可以用{\bf Schmidt正交化方法}构造出一列正规正交集$\{v_n\}$.操作如下:
        \begin{align*}
            v_1 &= \dfrac{u_1}{||u_1||}, \\
            v_2 &= \dfrac{u_2 - (u_2, v_1)v_1}{||u_2 - (u_2, v_1)v_1||}, \\
            v_3 &= \dfrac{u_3 - (u_3, v_1)v_1 - (u_3, v_2)v_2}{||u_3 - (u_3, v_1)v_1 - (u_3, v_2)v_2||}, \\
            &\vdots \\
            v_n &= \dfrac{u_n - \sum_{k=1}^{n-1} (u_n, v_k)v_k}{||u_n - \sum_{k=1}^{n-1} (u_n, v_k)v_k||}.
        \end{align*}
    \end{tcolorbox}
    \begin{definition}[正规正交基]
        设S是H中的正规正交集,如果H中没有其他的正规正交集真包含S,则称S为H的\textbf{正规正交基}
    \end{definition}
    他有一个等价叙述:\\
    \begin{proposition}
        设S是H中的正规正交集,则
        \center{S是H的正规正交基$\iff$H中没有非零元与S中的每个元正交}
    \end{proposition}
    关于正规正交基的存在性,我们有如下定理:
    \begin{itemize}
        \item 若H可分,则H中存在可数的正规正交基
        \item 每个非零的Hilbert空间都有正规正交基
        \item $\{e_\alpha \}$是H的一个正交基,$\alpha \in \mathcal{A}$是指标集,则对任意$x\in H$,都有
        \[x=\sum_{\alpha \in \mathcal{A}}(x,e_\alpha)e_\alpha\]
        \item {\bf Parseval等式:}$\forall x\in H$,都有
        \[\|x\|^2=\sum_{\alpha \in \mathcal{A}}|(x,e_\alpha)|^2\]
        在正交基下,向量的范数可以通过其在基上的投影来表示。
    \end{itemize}
    \begin{proposition}
        任何一个可分的Hilbert空间H都与$l^2$同构
    \end{proposition}
    \begin{tcolorbox}
        同构:存在双射$T:H\to l^2$,且$\forall x,y\in H$,都有
        \[(Tx,Ty)_{l^2}={(x,y)}_H\]
    \end{tcolorbox}

\section{射影定理、Fréchet-Riesz表示定理}

    \begin{definition}[正交补]
        设M是H的子空间(线性流形),则
        \[
            M^{\perp}=\{x\in H:(x,y)=0,\forall y\in M\}
        \]
        称$M^{\perp}$为M的\textbf{正交补} 
    \end{definition}
    显然,$M^{\perp}$是H的子空间,且$M\cap M^{\perp}=\{0\}$,而且$\overline{M}={(M^{\perp})}^{\perp}$\\
    下面是两个重要的定理:
    \begin{theorem}[射影定理]
        设H是Hilbert空间,M是H的子空间,则对任意$x\in H$,都可以唯一的表示为
        \[x=y+z,\quad y\in M,\ z\in M^{\perp}.\]
        y和z分别称为x在M上的\textbf{正交投影}和\textbf{正交余量}
    \end{theorem}
    应该指出的是,{\bf 射影定理实际上是Schmidt正交化的一个推广。}因为此时的M不再局限于有限维.\\
    射影定理使得Hilbert空间有着丰富的几何性质,从而区别于一般的Banach空间.\\

    \begin{definition}[对偶空间/共轭空间]
        设X是赋范线性空间,X上所有有界线性泛函$f:X\to \mathbb{C}$的全体构成的线性空间,记为$X^*$,称为X的\textbf{对偶空间}或\textbf{共轭空间}。\\
        对$f\in X^*$,定义
        \[
            ||f||=\sup_{||x||=1}|f(x)|
        \]
        则$||\cdot ||$是$X^*$上的范数,且$X^*$是Banach空间
        
    \end{definition}

    \begin{theorem}[Fréchet-Riesz定理]
        设H是Hilbert空间,$f:H\to \mathbb{R}$是连续线性泛函,则存在唯一的$y\in H$,使得
        \[f(x)={(x,y)}_H,\quad \forall x\in H.\]
        并且
        \[||f||_{H^*}=||y||_H.\]
    \end{theorem}

















    \end{document}